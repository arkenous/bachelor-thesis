%
% $Id: blank.tex,v 2.0 2010-01-05 18:50:50+09 kobayasi Exp $
%
% Mar 21, 2001:  Revision Control Started!!
%
\documentclass[11pt]{jreport}
\usepackage{newcent}             % PDFへの変換後の品質を高める
\usepackage[dvipdfmx]{graphicx}
\usepackage{color}
\definecolor{purple}{rgb}{0.6,0,0.4}
\definecolor{brown}{cmyk}{0,0.81,1,0.60}
\usepackage{listings, jlisting}
\renewcommand{\lstlistingname}{ソース}
\lstset
{
breaklines = true,
language=Java,
keywordstyle={\color{purple}},
commentstyle={\color{brown}},
numbers=left,
frame=single,
tabsize=4
}


%
%\usepackage[doctor]{gaiyo}      % 博士論文要旨の場合
%\usepackage[master]{gaiyo}      % 修士論文要旨の場合
\usepackage{sotsuron}               % 卒業研究概要の場合
% \usepackage[junior]{gaiyo}      % 専門演習レポートの場合

\setlength\textfloatsep{0pt}

\title{\bfseries スマートフォンのモーションセンサを利用した\\個人認証アプリケーションの開発}
\author{情11-0170 高坂 賢佑}
\date{}
\renewcommand{\bibname}{参考文献,参考URL等}

\begin{document}
\maketitle

\tableofcontents
\listoffigures

\chapter*{はじめに}
スマートフォンが徐々に普及しつつある現在,スマートフォンの個人認証方法は画面上に表示される
ソフトウェアキーボードのテンキーを用いたパスコード認証が大部分を占めている.しかし,この認証方法は
,画面ロックを解除するたびに画面に表示されたソフトウェアキーボードを目で見て指でタッチして操作する
必要が有るため,ユーザにとって煩雑な作業である.また,パスコード認証では,あらかじめ決められた文字種の
中から1つずつ選択したものを元にパスコードを構築していくという性質上,パターン数は限られ,自由度は
限定されてしまう.そこで,パスコード認証が抱える,認証の煩雑さを解消し,かつ,自由度が高くより直感的に
個人認証を行えるアプリケーションを開発する.このアプリケーションには,一般的なスマートフォンに
搭載されている加速度センサとジャイロセンサを用いることとする.

\chapter{使用するセンサについて}
	\section{加速度センサ}
	加速度センサとは,X軸,Y軸,Z軸の基準軸に対して直線運動の加速度をそれぞれ検出し,値として取り出すこと
	のできるセンサである.ここでいう加速度とは,端末における,単位時間あたりの速度の変化率のことを指し,
	における矢印の方向が正の値,逆が負の値をとる.
	
	\section{ジャイロセンサ}
	ジャイロセンサとは,X軸,Y軸,Z軸の基準軸に対して回転運動の角速度をそれぞれ検出し,値として取り出す
	ことのできるセンサである.ここでいう角速度とは,端末における,単位時間あたりの回転角のことを指し,
	右ねじを回した時に,における矢印の方向に進むような回転が正,逆が負の値をとる.

\chapter{先行研究}
	\section{坂本の研究}
	ddddddd
	\section{兎澤の研究}
	eeeeeeee
	\section{濱野,新井の研究}
	ffffffffff

\chapter{本研究のシステム}
	\section{本研究の概要}
	gggggggggg
	\section{システムの概要}
	hhhhhhhh
	\section{新規登録モード}
	iiiiiiiiiii
	\section{認証試験モード}
	jjjjjjjjjjjj
	\section{登録データ一覧モード}
	kkkkkkkkkkkkkkkk
\chapter{実験と考察}
	\section{実験方法}
	llllllllllll
	\section{実験結果}
	mmmmmmmmmmmmmmmm
	\section{考察}
	nnnnnnnnnnnnnnnn
	\section{課題}
	oooooooooooooooo
\chapter{おわりに}
pppppppppppppppp

\chapter*{謝辞}
eeeee

\end{document}